This dissertation is devoted to the dynamically developing field of ICT, which is cybersecurity. The work focuses on the vulnerability management process, in particular on methods that allow to improve the security of the ICT network by improving the vulnerability prioritization process, using knowledge regarding the monitored environment.

\bigbreak
The dissertation proposes methods of automatic vulnerability prioritization implementing the   generally available standards of criticality assessment and information regarding the importance of the asset from the organisation point of view. The aim of the work is to identify a group of critical, high and medium vulnerabilities that should be repaired in the first instance by adding an additional step in the phase of defining corrective actions in the vulnerability management process. In addition, the work includes mechanisms that allow to perform calculations for an increasing amount of data using scaling methods available in modern software development technologies.

\bigbreak
The analysis of the obtained results and the research carried out in real ICT environments confirmed the usefulness of the proposed algorithms. In addition, further research directions were proposed in the field of security analysis and monitoring of the investigated corporate networks with the use of information on known vulnerabilities.

\vspace*{1cm}
\textbf{Keywords:} vulnerability management process, known vulnerabilities, security of ICT infrastructure.
\pagestyle{empty}
Lista artykułów opublikowanych w trakcie realizacji pracy doktorskiej:
\begin{enumerate}
    \item {\emph{Referat konferencyjny}\newline
    Michał Walkowski, Jacek Oko, Sławomir Sujecki, Stanisław Kozdrowski\newline
    \textbf{The impact of cyber security on the quality of service in optical networks.}\newline
    Service Computation 2018 : The Tenth International Conferences on Advanced Service Computing [Dokument elektroniczny] : Barcelona, Spain, February 18-22, 2018 / eds. Arne Koschel, Janusz Klink, Andreas Hausotter. [B.m.] : IARIA, cop. 2018. s. 53-56.}
    \item {\emph{Referat konferencyjny} \newline
    Michał Walkowski, Maciej Biskup, Agata Szewczyk, Jacek Oko, Sławomir Sujecki\newline
    \textbf{Container based analysis tool for vulnerability prioritization in cyber security systems.}\newline
    2019 21st International Conference on Transparent Optical Networks (ICTON) / eds. Marek Jaworski, Marian Marciniak. Danvers, MA : IEEE ; Warsaw : National Institute of Telecommunications, cop. 2019. s. 1-4.}
    \item {\emph{Artykuł} - Punktacja z wykazu MNiSW: 100\newline
    Michał Walkowski, Maciej Krakowiak, Jacek Oko, Sławomir Sujecki\newline
    \textbf{Efficient algorithm for providing live vulnerability assessment in corporate network environment.}\newline
    Applied Sciences. 2020, vol. 10, nr 21, art. 7926, 1-16.}
    \item{\emph{Referat konferencyjny} - Punktacja z wykazu MNiSW: 70 \newline
    Michał Walkowski, Maciej Krakowiak, Jacek Oko, Sławomir Sujecki\newline
    \textbf{Distributed analysis tool for vulnerability prioritization in corporate networks.}\newline
    28th International Conference on Software, Telecommunications and Computer Networks, SoftCOM 2020, 17-19 September 2020, Hvar, Croatia. Croatia : FESB. University of Split, cop. 2020. s. 1-6.}
    \item{\emph{Referat konferencyjny} - Punktacja z wykazu MNiSW: 140\newline
    Maciej Nowak, Michał Walkowski, Sławomir Sujecki\newline
    \textbf{Machine learning algorithms for conversion of CVSS Base Score from 2.0 to 3.x.}\newline
    Computational Science - ICCS 2021 : 21st International Conference Krakow, Poland, June 16-18, 2021 : proceedings. Pt. 3 / eds. Maciej Paszynski [i in.]. Cham : Springer, cop. 2021. s. 255-269.}
    \item {\emph{Referat konferencyjny} - Punktacja z wykazu MNiSW: 70\newline
    Maciej Nowak, Michał Walkowski, Sławomir Sujecki\newline
    \textbf{Conversion of {CVSS} Base Score from 2.0 to 3.1}\newline
    The 29th International Conference on Software, Telecommunications and Computer Networks (SoftCOM 2021)}
    \item{\emph{Referat konferencyjny} - Punktacja z wykazu MNiSW: 70\newline
    Michał Walkowski, Maciej Krakowiak, Marcin Jaroszewski, Jacek Oko, Sławomir Sujecki\newline
    \textbf{Automatic {CVSS-based} vulnerability prioritization and response with context information}\newline
    The 29th International Conference on Software, Telecommunications and Computer Networks (SoftCOM 2021)}
    \item {\emph{Artykuł} - Punktacja z wykazu MNiSW: 100\newline
    Michał Walkowski, Jacek Oko, Sławomir Sujecki\newline
    \textbf{Vulnerability management models using Common Vulnerability Scoring System.}\newline
    Applied Sciences, 2021, vol. 11, nr 18, art. 8735, s. 1-25}
\end{enumerate}

\bigbreak
Wytworzone oprogramowanie w ramach pracy doktorskiej zostało zaakceptowane do fazy inkubator przez międzynarodową organizacje Open Web Application Security Project (OWASP). Jest to społeczność internetowa, która wytwarza ogólnodostępne artykuły, metodologie, dokumentację, narzędzia i technologie z zakresu bezpieczeństwa aplikacji internetowych. Faza inkubator reprezentuje projekty eksperymentalne, które wciąż są rozwijane. Strona projektu dostępna jest pod adresem: https://owasp.org/www-project-vulnerability-management-center.

\bigbreak
Całość wytworzonego oprogramowania udostępniono za pomocą platformy internetowej Github. W celu umożliwienia społeczności wzięcia udziału w projekcie oraz szerzenia dobrych praktyk z zakresu procesu zarządzania podatnościami. Kod aplikacji dostępny jest pod adresem: https://github.com/ DSecureMe/ vmc. Dokumentacja projektowa w języku angielskim: https://github.com/ DSecureMe/ vmc-docs oraz możliwość uruchomienia projektu w trybie demo: https://github.com/ DSecureMe/ vmc-demo.

\bigbreak
Dodatkowo przedstawione w niniejszej rozprawie opracowane oprogramowanie zostało, wdrożone i przetestowane w ramach projektu badawczego RegSoc realizowanego przez Wrocławskie Centrum Sieciowo-Superkomputerowe Politechniki Wrocławskiej, którego celem jest podniesienie poziomu bezpieczeństwa cyfrowego w sektorze publicznym.

\bigbreak
Projekt RegSoc realizowany jest w partnerstwie z Instytutem Technik Innowacyjnych EMAG oraz Państwowym Instytutem Badawczym NASK, dedykowany dla administracji rządowej i samorządowej oraz jednostkom akademickim. Docelowo będzie mógł być wykorzystywany również przez podmioty niepubliczne jako rozwiązanie alternatywne dla systemów komercyjnych.

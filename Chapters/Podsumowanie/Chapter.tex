W ramach przeprowadzonych prac badawczych dokonano implementacji modelu zarządzania podatnościami przedstawionego na rysunku \ref{fig:chapter1:vm-model-cvss2e}, który do przeprowadzenia prioretyzacji podatności oprócz oceny bazowej CVSS 2.0 wykorzystuje dane środowiskowe. Opracowany system zarządzania podatnościami ma jednak istotne wady. Dlatego też dokonano implementacji modelu zarządzania podatnościami przedstawionego na rysunku \ref{fig:chapter1:vm-model-cvss3e}, który oprócz oceny bazowej CVSS 3.x korzysta z danych środowiskowych. Implementacja modelu zarządzania podatnościami przedstawionego na rysunku \ref{fig:chapter1:vm-model-cvss3e} była możliwa dzięki zastosowaniu mechanizmów uczenia maszynowego  \cite{Nowak-cldd-2021, Nowa2109Conversion} do konwersji oceny bazowej CVSS 2.0 do standardu CVSS 3.x. W opracowanym pakiecie oprogramowania wykorzystano ponadto konteneryzację w celu zmniejszenia czasu trwania obliczeń ocen środowiskowych.

\bigbreak
Na podstawie analizy wyników stwierdza się, że opracowane oprogramowanie w ramach rozprawy doktorskiej z wykorzystaniem najnowszych technologii, takich jak Docker oraz Kubernetes, przystosowane jest do przetwarzania przyrastającej ilości danych. Otrzymane czasy obliczeń dla wszystkich rozpatrywanych przypadków w porównaniu do czasu trwania skanowania oraz naprawy podatności są pomijalne, ponieważ czas skanowania oraz naprawy podatności liczony jest w godzinach. Na przykład najkrótszy czas skanowania dla środowiska teleinformatycznego A wynosi 3 godziny. Natomiast czas naprawy podatności mieści się w zakresie od 1 do 9 roboczogodzin \cite{farris2018vulcon}. Ponadto opracowane oprogramowanie w przypadku zwiększenia liczby aktywnych modułów obliczeniowych o jeden pozwala na zmniejszenie czasu przetwarzania danych aż o 45\%. W pełni zatem uzasadnione jest pominięcie w równaniach \ref{eq:cvss2e}, \ref{eq:cvss3e} czasu trwania wykonywania obliczeń ($T_{VMC}$). Otrzymane wyniki pozwalają na stwierdzenie, że wykorzystując wytworzone oprogramowanie, możliwe jest rozwiązanie problemu przedstawionego w rozdziale \ref{sec:modele-zarzadzaia-podatnosciami}, dotyczącego skalowalności. Mianowicie wytworzone oprogramowanie w ramach rozprawy doktorskiej dostosowane jest do rosnącej ilości napływających danych i nie ma wpływu na czas pozostawania podatności bezpieczeństwa w infrastrukturze teleinformatycznej.

\bigbreak
Na podstawie otrzymanych wyników można stwierdzić, że parametry środowiskowe $CIA$, $CDP$, $TD$ mają znaczący wpływ na zmianę kategorii krytyczności poprzez modyfikacje wartości oceny bazowej CVSS 2.0. Jak wykazano w przeprowadzonej analizie, istnieje 540 możliwych kombinacji parametrów środowiskowych, które przekładają się na możliwe do uzyskania wartości oceny środowiskowej CVSS 2.0. Dla podatności o kategorii wysokiej z oceną bazową CVSS 2.0 o wartości 7.5 istnieje 55 unikalnych wartości oceny środowiskowej, pozwalającej na zmianę kategorii z wysokiej na średnią lub niską z prawdopodobieństwem: wysoka - 33.3\%, średnia - 35.1\%, niska - 31.6\%. Dla podatności o kategorii średniej z oceną bazową CVSS 2.0 o wartości 4.3 istnieje 30 unikalnych wartości oceny środowiskowej, pozwalającej na zmianę kategorii z średniej na niską lub wysoką z prawdopodobieństwem: wysoka - 10.6\%, średnia - 46.9\%, niska - 42.5\%. Dla podatności o kategorii niskiej z oceną bazową CVSS 2.0 o wartości 3.3 istnieje 47 unikalnych wartości oceny środowiskowej, pozwalającej na zmianę kategorii z niskiej na wysoką lub średnią z prawdopodobieństwem: wysoka - 2.2\%, średnia - 41.5\%, niska - 56.3\%. W związku z czym prawidłowym stwierdzeniem jest, że zmiana jednego parametru środowiskowego $CIA$, $CDP$, $TD$ może wpłynąć na zmianę kategorii krytyczności podatności. Dodatkowo wykorzystanie parametrów środowiskowych $CIA$, $CDP$, $TD$ wpływa na lepsze dopasowanie oceny środowiskowej CVSS do wymagań organizacji \cite{fruhwirth2009improving, wang2015vulnerability, gallon2010impact}. Dotychczas jednak,  według najlepszej wiedzy autora, brak było rozwiązań, które obliczałyby w sposób automatyczny ocenę środowiskową CVSS 2.0, dlatego też model zarządzania podatnościami oparty o ocenę środowiskową nie był wykorzystywany. Wpływ na to ma ilość otrzymywanych danych ze skanerów podatności oraz liczba monitorowanych systemów, która powoduje, że niemożliwe jest wykonanie obliczeń dla oceny środowiskowej w sposób ręczny. Na przykład dla analizowanego środowiska C analityk bezpieczeństwa musiałby wykonać obliczenia dla 10 078 podatności mając do wyboru 540 możliwych kombinacji parametrów środowiskowych oceny CVSS 2.0. Wykonanie obliczeń w sposób ręczny przez analityka bezpieczeństwa spowodowałoby znaczne wydłużenie liczby roboczogodzin wymaganych do usunięcia istotnych podatności środowisk teleinformatycznych. Dodatkowo wykonanie obliczeń w sposób ręczny może wpłynąć na pojawianie się błędów w otrzymanych ocenach i kategoriach podatności, co również w konsekwencji będzie miało wpływ na czas ekspozycji systemu na zagrożenia wynikające z ataków hakerskich. 
Dlatego też w ramach niniejszej rozprawy doktorskiej, napisano oprogramowanie, które integruje się ze skanerami podatności oraz bazami zasobów w celu obliczenia wartości oceny środowiskowej CVSS 2.0. Następnie dodano do oceny bazowej CVSS 2.0 dane środowiskowe, dzięki czemu możliwe było zaimplementowanie modelu zarządzania podatnościami przedstawionego na rysunku \ref{fig:chapter1:vm-model-cvss2e} we wszystkich analizowanych środowiskach teleinformatycznych. Na podstawie otrzymanych wyników można stwierdzić, że uwzględnienie danych środowiskowych $CIA$, $CDP$, $TD$ ma znaczący wpływ na zmiany w kategoriach krytyczności wykrytych podatności dla wszystkich analizowanych środowisk teleinformatycznych. Dodatkowo zmiany w kategoriach krytyczności podatności przełożyły się na priorytetyzacje napraw podatności, a tym samym na szacunkową liczbę roboczogodzin wymaganą do usunięcia istotnych podatności bezpieczeństwa. Na podstawie wyników otrzymanych dla środowisk teleinformatycznych A i C można stwierdzić, że dla oceny środowiskowej CVSS 2.0 wszystkie podatności zostały przydzielone do kategorii niskiej. Wpływ na to ma parametr $TD$, który osiągnął maksymalną wartość dla środowiska teleinformatycznego A - 13\%, dla środowiska teleinformatycznego C - 3\%. Dla środowiska teleinformatycznego B większość podatności została sklasyfikowana do kategorii niskiej. Wpływ na to ma parametr $TD$, który osiągnął maksymalną wartość 25\%, co oznacza oznacza że w środowisku teleinformatycznym B wykryte zostały podatności, na które wrażliwe jest 9 skanowanych zasobów. Dlatego też podatności, dla których parametr $TD$ ma wartość 25\%, zostały przydzielone do kategorii średniej. Dodatkowo na podstawie wyników można stwierdzić, że dodanie do oceny bazowej CVSS 2.0 danych środowiskowych powoduje znaczącą redukcję szacowanej liczby roboczogodzin wymaganej do usunięcia istotnych podatności bezpieczeństwa. Dla środowiska teleinformatycznego A na podstawie otrzymanych wyników można zauważyć, że zysk w postaci wykorzystania modelu zarządzania podatnościami opartego o ocenę środowiskową CVSS 2.0 znacząco redukuje szacowaną liczbę roboczogodzin o 96.9\% w porównaniu do wymaganej liczby roboczogodzin otrzymanej dla modelu zarządzania podatnościami opartego o ocenę bazową CVSS 2.0. Ponadto dla środowiska teleinformatycznego A wykorzystanie oceny środowiskowej CVSS 2.0 pozwala na obniżenie ryzyka związanego z nienaprawieniem podatności niedoszacowanej z 10\% do 0\%. Dla środowiska teleinformatycznego B na podstawie otrzymanych wyników można zauważyć, że zysk w postaci wykorzystania modelu zarządzania podatnościami opartego o ocenę środowiskową CVSS 2.0 znacząco redukuje szacowaną liczbę roboczogodzin o 75.9\% w porównaniu do wymaganej liczby roboczogodzin otrzymanej dla modelu zarządzania podatnościami opartego o ocenę bazową CVSS 2.0. Dodatkowo dla środowiska teleinformatycznego B wykorzystanie oceny środowiskowej CVSS 2.0 pozwala na obniżenie ryzyka związanego z nienaprawieniem podatności niedoszacowanej z 3.8\% do 0\%. Dla środowiska teleinformatycznego C na podstawie otrzymanych wyników można zauważyć, że zysk w postaci wykorzystania modelu zarządzania podatnościami opartego o ocenę środowiskową CVSS 2.0 znacząco redukuje szacowaną liczbę roboczogodzin o 99.9\% w porównaniu do wymaganej liczby roboczogodzin otrzymanej dla modelu zarządzania podatnościami opartego o ocenę bazową CVSS 2.0. Oprócz tego dla środowiska teleinformatycznego C wykorzystanie oceny środowiskowej CVSS 2.0 pozwala na obniżenie ryzyka związanego z nienaprawieniem podatności niedoszacowanej z 17.8\% do 0\%. Podsumowując, wykorzystanie oceny środowiskowej CVSS 2.0 oraz wytworzonego oprogramowania na potrzeby niniejszej rozprawy znacząco redukuje wymaganą liczbę roboczogodzin, potrzebną do usunięcia istotniej podatności środowiska teleinformatycznego. Natomiast przeprowadzona analiza otrzymanych wyników potwierdza również wadę oceny środowiskowej CVSS 2.0 polegającą na tym, że parametr $TD$, który służy do ustalania liczby systemów wrażliwych na daną podatność, znacząco zaniża oceny wszystkich podatności. Pomimo wspomnianej wady, na podstawie otrzymanych wyników, można dodatkowo stwierdzić, że wartość parametru $TD$ może być wykorzystana do wykrywania podatności, która może zagrozić większej liczbie skanowanych zasobów. Dlatego możliwe jest stwierdzenie, że wykorzystanie danych na temat infrastruktury teleinformatycznej oraz danych pochodzących ze skanerów podatności umożliwia w oparciu o otwarty standard CVSS 2.0 zwiększyć dokładność priorytetyzacji podatności bezpieczeństwa infrastruktury teleinformatycznej, a tym samym zmniejszyć liczbę roboczogodzin potrzebną do usunięcia istotnych podatności. Ponieważ jednak parametr $TD$ zaniża oceny wszystkich podatności oraz standard CVSS 3.x lepiej ocenia istotę podatności i skuteczniej szacuje zagrożenia \cite{fall2019common, Nowak-cldd-2021, Nowa2109Conversion} w porównaniu do oceny bazowej CVSS 2.0, konieczne było rozważenie modelu zarządzania podatnościami, który opiera się na standardzie CVSS 3.x. 

\bigbreak
Na podstawie analizy wyników dotyczących standardu CVSS 3.x stwierdza się, że parametry środowiskowe $CIA$ mają znaczący wpływ na zmianę kategorii krytyczności poprzez modyfikacje wartości oceny bazowej CVSS 3.x. Jak wykazano w przeprowadzonej analizie, istnieje 27 możliwych kombinacji parametrów środowiskowych, które przekładają się na możliwe do uzyskania wartości oceny środowiskowej CVSS 3.x. Dla podatności o kategorii krytycznej z oceną bazową CVSS 3.x o wartości 9.8 istnieją 4 unikalne wartości oceny środowiskowej, pozwalającej na zmianę kategorii z krytycznej na wysoką z prawdopodobieństwem: wysoka - 3.7\%, krytyczna - 96.3\%. Dla podatności o kategorii wysokiej z oceną bazową CVSS 3.x o wartości 7.5 istnieją 3 unikalne wartości oceny środowiskowej, pozwalającej na zmianę kategorii z wysokiej na średnią lub krytyczną z prawdopodobieństwem: krytyczna - 33.(3)\%, wysoka - 33.(3)\%, średnia - 33.(3)\%. Dla podatności o kategorii średniej z oceną bazową CVSS 3.x o wartości 4.3 istnieją 3 unikalne wartości oceny środowiskowej, pozwalającej na zmianę kategorii z średniej na niską z prawdopodobieństwem: średnia - 66.(6)\%, niska - 33.(3)\%. Dla podatności o kategorii niskiej z oceną bazową CVSS 3.x o wartości 3.6 istnieje 6 unikalnych wartości oceny środowiskowej, pozwalającej na zmianę kategorii z niskiej na średnią z prawdopodobieństwem: niska - 66.(6)\%, średnia - 33.(3)\%. W związku z czym, zasadnym jest stwierdzenie, że zmiana jednego parametru środowiskowego $CIA$ może wpłynąć na zmianę kategorii krytyczności podatności. Dodatkowo wykorzystanie parametrów środowiskowych $CIA$ wpływa na lepsze dopasowanie oceny środowiskowej CVSS do wymagań organizacji \cite{fruhwirth2009improving, wang2015vulnerability, gallon2010impact}. Natomiast nie wszystkie publiczne znane podatności mają ocenę bazową według standardu CVSS 3.x. Wynika to z dużej luki czasowej pomiędzy publikacją standardów CVSS 2.0 i CVSS 3.x, dużej liczby wykrywanych i opublikowanych podatności oraz istotnymi różnicami w sposobie określania właściwości wektorowych oceny bazowej. Na przykład ocena bazowa CVSS 2.0 opisana jest 6 parametrami, z których każdy może przyjmować jedną z trzech wartości, natomiast ocena bazowa CVSS 3.x - 8 parametrami, z których każdy może przyjmować jedną z kilku wartości (od 2 do 4). Z kolei w celu określenia prawidłowej wartości wektorowej oceny bazowej analityk bezpieczeństwa musi zapoznać się szczegółowo z każdą podatnością. W przypadku środowiska teleinformatycznego A, dla którego brakuje oceny bazowej CVSS 3.x dla 34\% wykrytych podatności, zadanie staje się trudne oraz czasochłonne. Ostatecznie wszystkie błędy w ręcznym wykonaniu ewaluacji oceny bazowej CVSS 3.x przełożą się na czas istnienia podatności w systemie. Problem braku oceny bazowej CVSS 3.x dla wszystkich wykrytych podatności został rozwiązany za pomocą mechanizmu uczenia maszynowego, który pozwolił na konwersje oceny bazowej CVSS 2.0 do 3.x. Dzięki zastosowaniu mechanizmu uczenia maszynowego możliwa była pełna implementacja modelu zarządzania podatnościami opartego na standardzie CVSS 3.x. Kolejnym problemem to, według najlepszej wiedzy autora, brak rozwiązań, które pozwalałby w sposób automatyczny na obliczanie oceny środowiskowej CVSS, dlatego też model zarządzania podatnościami oparty o ocenę środowiskową CVSS 3.x nie był wykorzystywany. Wpływ na to ma również ilość otrzymywanych danych ze skanerów podatności oraz liczba monitorowanych systemów, która powoduje, że niemożliwe jest wykonanie obliczeń dla oceny środowiskowej w sposób ręczny. Na przykład dla analizowanego środowiska A analityk bezpieczeństwa musiałby wykonać obliczenia dla 10 078 podatności, mając do wyboru 27 możliwych kombinacji parametrów środowiskowych oceny CVSS 3.x. Wykonanie obliczeń w sposób ręczny przez analityka bezpieczeństwa, spowodowałoby znaczne wydłużenie liczby roboczogodzin wymaganych do usunięcia istotnych podatności środowisk teleinformatycznych. Natomiast wykonanie obliczeń w sposób ręczny może wpłynąć na pojawianie się błędów w otrzymanych ocenach i kategoriach podatności, co również w konsekwencji będzie miało wpływ na czas ekspozycji systemu na zagrożenia wynikające z ataków hakerskich. Dlatego też, w ramach niniejszej rozprawy doktorskiej, napisane zostało oprogramowanie, które integruje się ze skanerami podatności oraz bazami zasobów. Następnie dla środowisk teleinformatycznych A i C dodano do oceny bazowej CVSS 3.x dane środowiskowe, które pozwoliły na implementacje modelu zarządzania podatnościami przedstawionego na rysunku \ref{fig:chapter1:vm-model-cvss3e}. Dla środowiska teleinformatycznego B nie można było zaimplementować modelu zarządzania podatnościami opartego na ocenie środowiskowej CVSS 3.x, ponieważ administrator środowiska teleinformatycznego B nie dostarczył informacji na temat parametrów środowiskowych $CIA$. Na podstawie wyników można stwierdzić, że wykorzystany mechanizm uczenia maszynowego wykonał obliczenia dla: 
\begin{itemize}
    \item środowiska teleinformatycznego A, konwertując 34\% podatności, które nie posiadają oceny bazowej CVSS 3.x z błędem klasyfikacji wynoszącym 14\%, co oznacza, że można oczekiwać maksymalnie 2 podatności błędnie sklasyfikowanych,
    \item środowiska teleinformatycznego C, konwertując 3.45\%, które nie posiadają oceny bazowej CVSS 3.x z błędem klasyfikacji wynoszącym 14\%, co oznacza, że można oczekiwać maksymalnie 48 podatności błędnie sklasyfikowanych.
\end{itemize}
Na podstawie analizy wyników można także stwierdzić, że parametry środowiskowe $CIA$ mają wpływ na zmianę kategorii krytyczności. W środowisku teleinformatycznym A wpływ parametrów $CIA$ spowodował zmianę kategorii krytyczności dla 24.39\% podatności. W środowisku teleinformatycznym C wpływ parametrów $CIA$ spowodował zmianę kategorii krytyczności dla 20.30\% podatności. Zmiana kategorii krytyczności dla obu środowisk teleinformatycznych A i C ma bezpośredni przekład na szacowaną liczbę roboczogodzin potrzebną do usunięcia istotnych podatności bezpieczeństwa. Gdy porównamy wyniki otrzymane dla środowiska teleinformatycznego A i oceny bazowej CVSS 3.x bez zastosowania metod konwersji oceny bazowej CVSS 2.0 do 3.x ($T_{Base3}$) z oceną środowiskową CVSS 3.x z zastosowaniem metod konwersji oceny bazowej CVSS 2.0 do 3.x ($T_{Env3ML}$), można zauważyć wzrost szacowanej liczby roboczogodzin o 47.6\%. Wzrost szacowanej liczby roboczogodzin wynika z uwzględnienia wszystkich podatności po zastosowaniu metod konwersji oceny bazowej CVSS 2.0 do 3.x za pomocą uczenia maszynowego. Uwzględnienie wszystkich podatności i wykonanie priorytetyzacji napraw za pomocą oceny środowiskowej CVSS 3.x z wykorzystaniem metod konwersji oceny bazowej CVSS 2.0 do 3.x redukuje ryzyko dotyczące nienaprawienia podatności, która może zostać wykorzystana przez atakującego z 34.1\% do 2.4\%, przez co zmniejsza czas obecności podatności w systemie. Gdy porównamy wyniki otrzymane dla środowiska teleinformatycznego C i oceny bazowej CVSS 3.x ($T_{Base3ML}$) z oceną środowiskową CVSS 3.x ($T_{Env3ML}$), można zauważyć redukcję szacowanej liczby roboczogodzin o 10.4\%. Redukcja szacowanej liczby roboczogodzin wynika z dopasowania kategorii podatności do monitorowanego środowiska teleinformatycznego C (Rozdział \ref{sec:modele-zarzadzaia-podatnosciami}) oraz mniejszej liczby podatności, które musiały zostać przekonwertowane z oceny bazowej CVSS 2.0 do 3.x. Wykonanie priorytetyzacji napraw podatności za pomocą oceny środowiskowej CVSS 3.x z wykorzystaniem metod konwersji oceny bazowej CVSS 2.0 do 3.x dla środowiska teleinformatycznego C redukuje ryzyko dotyczące nienaprawienia podatności, która może zostać wykorzystana przez atakującego z 9.9\% do 0.5\%. Podsumowując, wykorzystanie oceny środowiskowej CVSS 3.x oraz wytworzonego oprogramowania na potrzeby niniejszej rozprawy doktorskiej wpływa pozytywnie na zmniejszenie wymaganej liczby roboczogodzin do usunięcia istotniej podatności środowiska teleinformatycznego, jednocześnie zmniejszając ryzyko nienaprawienia podatności z powodu braku oceny bazowej CVSS 3.x, która może zostać wykorzystana przez atakującego. Na podstawie przedstawionych wyników stwierdza się, że wykorzystanie danych na temat infrastruktury teleinformatycznej oraz danych pochodzących ze skanerów podatności umożliwia w oparciu o otwarty standard CVSS 3.x zwiększyć dokładność priorytetyzacji podatności bezpieczeństwa infrastruktury teleinformatycznej, a tym samym zmniejszyć liczbę roboczogodzin potrzebną do usunięcia istotnych podatności.

\bigbreak
Wytworzone oprogramowanie w ramach pracy doktorskiej zostało zaakceptowane do fazy inkubator przez międzynarodową organizacje Open Web Application Security Project (OWASP). Jest to społeczność internetowa, która wytwarza ogólnodostępne artykuły, metodologie, dokumentację, narzędzia i technologie z zakresu bezpieczeństwa aplikacji internetowych. Faza inkubator reprezentuje projekty eksperymentalne, które wciąż są rozwijane. Strona projektu dostępna jest pod adresem: https://owasp.org/www-project-vulnerability-management-center.

\bigbreak
Całość wytworzonego oprogramowania udostępniono za pomocą platformy internetowej Github. W celu umożliwienia społeczności wzięcia udziału w projekcie oraz szerzenia dobrych praktyk z zakresu procesu zarządzania podatnościami. Kod aplikacji dostępny jest pod adresem: https://github.com/ DSecureMe/ vmc. Dokumentacja projektowa w języku angielskim: https://github.com/ DSecureMe/ vmc-docs oraz możliwość uruchomienia projektu w trybie demo: https://github.com/ DSecureMe/ vmc-demo.

\bigbreak
Dodatkowo przedstawione w niniejszej rozprawie opracowane oprogramowanie zostało wdrożone i przetestowane w ramach projektu badawczego RegSoc realizowanego przez Wrocławskie Centrum Sieciowo-Superkomputerowe Politechniki Wrocławskiej, którego celem jest podniesienie poziomu bezpieczeństwa cyfrowego w sektorze publicznym.

\bigbreak
Projekt RegSoc realizowany jest w partnerstwie z Instytutem Technik Innowacyjnych EMAG oraz Państwowym Instytutem Badawczym NASK, dedykowany dla administracji rządowej i samorządowej oraz jednostkom akademickim. Docelowo będzie mógł być wykorzystywany również przez podmioty niepubliczne jako rozwiązanie alternatywne dla systemów komercyjnych.

\bigbreak
Najbardziej obiecujący kierunek dalszych badań związany jest z opracowaniem  metod dokładniejszego obliczania oceny środowiskowej CVSS 3.x. W szczególności  opracowanie praktycznie użytecznej metody obliczania oceny czasowej CVSS 2.0 oraz CVSS 3.x. 

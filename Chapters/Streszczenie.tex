Niniejsza rozprawa poświęcona jest ostatnio dynamicznie rozwijającej się dziedzinie teleinformatyki, jaką jest cyberbezpieczeństwo. W pracy skoncentrowano się na procesie zarządzania podatnościami, a w szczególności na metodach, które pozwalają na poprawienie bezpieczeństwa sieci teleinformatycznej poprzez udoskonalenie procesu priorytetyzacji podatności, wykorzystując wiedzę o monitorowanym środowisku.

\bigbreak
W rozprawie zaproponowane zostały metody automatycznej priorytetyzacji podatności z wykorzystaniem ogólnodostępnych standardów oceny krytyczności oraz informacji o wadze zasobu z punktu widzenia organizacji. Celem pracy jest wyznaczenie grupy podatności krytycznych, wysokich oraz średnich, jak również określenie kolejności, w jakiej powinny one być naprawiane poprzez dodanie kroku w fazie definiowania działań naprawczych procesu zarządzania podatnościami. Dodatkowo w pracy zostały ujęte mechanizmy pozwalające na wykonywanie obliczeń dla przyrastającej ilości danych z wykorzystaniem metod skalowania dostępnych w nowoczesnych technologiach rozwoju oprogramowania.

\bigbreak
Analiza uzyskanych wyników oraz badania przeprowadzone w rzeczywistych środowiskach teleinformatycznych potwierdziły przydatność zaproponowanych algorytmów. Ponadto, zaproponowano dalsze kierunki badań w zakresie analizy bezpieczeństwa oraz monitoringu badanych sieci korporacyjnych, przy wykorzystaniu informacji o znanych podatnościach.

\vspace*{1cm}
\textbf{Słowa kluczowe:} proces zarządzania podatnościami, znane podatności, bezpieczeństwo infrastruktury teleinformatycznej.